% EVERY SECTION FILE SHOULD BE NAMED NN_sectionname.tex
% where NN is order number with leading zeros
%
% 00_introduction.tex is reserved for the intro-text of the whole report. 

% Name of section is start of new part of assignment
\section{SQL}
\texttt{SQL}, or \textit{Sequential Query Language}, is a language \dots

%%%%%%%%%%%%%%%%%%%%%%%%%%%%%%%%%%%%%%%%%%%%%%%%%%%%%%%%%%
\subsection{Example statements}

\subsubsection{Functioning SQL statements}
Using the SQL Tryit Editor\footnote{http://www.w3schools.com/sql/trysql.asp?filename=trysql\_select\_all}
by w3schools \dots

\textbf{Create a list of customers and their contact person}
\begin{lstlisting}
SELECT CustomerName, ContactName
FROM Customers
WHERE ContactName IS NOT NULL
ORDER BY ContactName DESC
LIMIT 0,5
\end{lstlisting}

\textbf{Extract all suppliers who carry more than 4 products}
\begin{lstlisting}
SELECT SupplierName, COUNT(DISTINCT ProductID) AS NumberOfProducts
FROM Suppliers s
LEFT JOIN Products p
  ON s.SupplierID = p.SupplierID
GROUP BY SupplierName
HAVING NumberOfProducts > 4
LIMIT 0,5
\end{lstlisting}


\subsubsection{Non-Functioning SQL statements}
\textbf{Select a random number of customers (at probability 0.5) from the
  database}
\textit{Error 1: could not prepare statement (1 not authorized to use function: random)}
\begin{lstlisting}
SELECT *
FROM Customers
WHERE RANDOM() > 0.5
ORDER BY CustomerName
\end{lstlisting}

\textbf{Select all customers that have placed at least 5 orders}
\textit{Error 1: could not prepare statement (1 misuse of aggregate function COUNT())}
\begin{lstlisting}
SELECT DISTINCT *
FROM Customers c
LEFT JOIN Orders o
ON c.CustomerID = o.CustomerID
  WHERE COUNT(*) > 4
GROUP BY c.CustomerID 
\end{lstlisting}
