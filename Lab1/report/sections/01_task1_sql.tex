% EVERY SECTION FILE SHOULD BE NAMED NN_sectionname.tex
% where NN is order number with leading zeros
%
% 00_introduction.tex is reserved for the intro-text of the whole report. 

% Name of section is start of new part of assignment
\section{Task 1: SQL}
\texttt{SQL}, or \textit{Structured Query Language}, is a language designed for
managing data stored in relational database systems. SQL exists under many names,
some which have extended the SQL with further functionality, such as MySQL, MSSQL,
SQLite or WebSQL.

Relational database systems are popular and their use wide-spread -- many websites
use a relational database system based on SQL to store all data shown- and handled
by the website. One of the reasons SQL is important in the field of information
security is that the vulnerability known as SQL injection, SQLi for short, is still
one of the most popular vulnerabilities to exploit\cite{MOTHBRDSQL}. This even though
the vulnerabilty was disclosed more than 15 years ago.

In this task, four examples of SQL statements (also referred to as SQL queries or queries)
-- that is queries written in SQL form to be processed by a SQL server of some kind --
are explained, of which two function correctly and two are erroneous.

%%%%%%%%%%%%%%%%%%%%%%%%%%%%%%%%%%%%%%%%%%%%%%%%%%%%%%%%%%
\subsection{Example statements}

The examples are written for WebSQL, a client-side database\cite{W3CWEBSQL}
(for example a web browser), which is based on the dialect of SQLite 3\cite{W3CWEBSQL}.
The SQL Tryit Editor\footnote{http://www.w3schools.com/sql/trysql.asp?filename=trysql\_select\_all}
by w3schools can be used to test SQL queries against the WebSQL interface.

\subsubsection{Part A: Functioning SQL statements}
The following SQL statements are fully functional (non erroneous) when processed
by a WebSQL database. 

\paragraph{Create a list of customers and their contact person}

\begin{lstlisting}[caption={Query for retreiving the name of five customers and their contact person},label={sqlstatement1}]
SELECT CustomerName, ContactName
FROM Customers
WHERE ContactName IS NOT NULL
ORDER BY ContactName DESC
LIMIT 0,5
\end{lstlisting}

\begin{table}[h]
  \centering
  \begin{tabular}{ | l | l | }
    \hline
    \textbf{CustomerName} & \textbf{ContactName} \\
    \hline
    Wolski & Zbyszek  \\
    \hline
    Océano Atlántico Ltda. & Yvonne Moncada \\
    \hline
    Laughing Bacchus Wine Cellars & Yoshi Tannamuri \\
    \hline
    Hungry Coyote Import Store & Yoshi Latimer \\
    \hline
    Chop-suey Chinese & Yang Wang \\
    \hline
  \end{tabular}
  \caption{Result from SQL query in Listing \ref{sqlstatement1}}
  \label{table:sql1}
\end{table}


This SQL query is designed to work with a database that contains a table
called \textbf{Customers}, in which there are two columns called
\textbf{CustomerName} and \textbf{ContactName}. This can be derived
from the first two rows of the statement; it \texttt{SELECT}s the
columns \textbf{CustomerName} and \textbf{ContactName} \texttt{FROM}
the table \textbf{Customers}.

The query also filters the results so that only rows \texttt{WHERE}
\textbf{CustomerName} has been set is included, in other words where
\textbf{CustomerName} \texttt{IS NOT NULL}\footnote{\texttt{NULL} in SQL is a
  value denoting not knowing the actual value in the selected row and column}.
It then sorts the results on \textbf{ContactName} in \texttt{DESC}ending order,
as denoted by the \texttt{ORDER BY} clause. Lastly it returns at most five (5)
results, where \texttt{LIMIT 0,5} denotes only five rows starting from row 0
should be returned.


\paragraph{Extract all suppliers who carry more than 4 products}
For this query, only concepts not explained in the prior query will be
explained in detail.

\begin{lstlisting}[caption={SQL Query to retreive all suppliers carrying more
      than four products.},label={sqlstatement2}]
SELECT SupplierName, COUNT(DISTINCT ProductID) AS NumberOfProducts
FROM Suppliers s
LEFT JOIN Products p
  ON s.SupplierID = p.SupplierID
GROUP BY SupplierName
HAVING NumberOfProducts > 4
\end{lstlisting}

\begin{table}[h]
  \centering
  \begin{tabular}{ | l | l | }
    \hline
    \textbf{SupplierName} & \textbf{NumberOfProducts} \\
    \hline
    Pavlova, Ltd. & 5  \\
    \hline
    Plutzer Lebensmittelgro{\ss}märkte AG & 5 \\
    \hline
  \end{tabular}
  \caption{Result from SQL query in Listing \ref{sqlstatement2}}
  \label{table:sql2}
\end{table}


This SQL query is designed to work with a database that contains a table
called \textbf{Suppliers} and a table called \textbf{Products}. In the
\texttt{SELECT}-row of the query, an aggregate value, \texttt{COUNT} can be
seen. An aggregate value is a value calculated during the retreival of the
result-set. In this case \texttt{COUNT} counts the number of unique
values in the column \textbf{ProductID}. What makes the query only count
the unique values (eg. a set of $\{1,2,2,3\} would count only three unique
values$) is the keyword \texttt{DISTINCT}. This aggregate value is then
aliased (given a name) \textbf{NumberOfProducts} for ease-of-use as well
as recognisability.

As the data used in this query is located in two tables, one being
\textbf{Suppliers} of the \texttt{SELECT}-part of the query, both
tables need to be linked together in some meaningful way. This is
accomplished using \texttt{JOIN}. In this query, rows from the table
\textbf{Suppliers} are linked to rows in the table \textbf{Products}
by matching the value of the column \textbf{SupplierID} in both
tables (using the keyword \texttt{ON}). Thus each row won't be
randomly matched to another row in the second table. The \texttt{LEFT}
keyword in \texttt{LEFT JOIN} assures that all rows in the first table
(\textbf{Suppliers}) will be returned even if there are no matching
rows in the secondary table (\textbf{Products}). The reverse is however
not true.

The query also \texttt{GROUP}s all retreived rows by the value in the
column \textbf{SupplierName}. This is important, otherwise the result
would only consist of a single row as the aggregate value (\texttt{COUNT})
then would range over all rows and not only the rows of the individual
suppliers.

The last row of the query filters the results, only returning rows
where the column containing the aggregate value is higher than four
(eg. \texttt{COUNT}ed unique -- for the supplier -- products are more
than four). The \texttt{HAVING} keyword may seem analogous to \texttt{WHERE},
as both filter the results of a query. There is, however, a fundamental
difference between the both; \texttt{WHERE} filters the results \textit{during}
retreival of the result-set, \texttt{HAVING} filters \textit{the retrieved}
result-set (in other words after the results have been retrieved). Thus \texttt{WHERE}
cannot be applied to aggregate values, as they are calculated \textit{during} the
retrieval, but \texttt{HAVING} can.

\subsubsection{Part B: Non-Functioning SQL statements}
The following SQL statements are erroneous when processed by a WebSQL database.
As all of the concepts and keywords have been explained in the previous
section, the main focus will be what the queries were ment to do, what is
wrong with the query and why it is erroneous.

\paragraph{Select a random number of customers (at probability 0.5) from the
  database}

\begin{lstlisting}[label={sqlstatement3}]
SELECT *
FROM Customers
WHERE RANDOM() > 0.5
ORDER BY CustomerName
\end{lstlisting}

This query is designed to retreive a random amount of customers with each
use. As mentioned previously, the WebSQL database is based on SQLite 3,
which has a function called \texttt{RANDOM}. The \texttt{RANDOM} function
is designed to return a random value between 0 and 1. As the function is used
in the \texttt{WHERE} clause of the query, we use it to filter the result
based on randomness.

For each row retreived, a random value is generated and the inequality is
evaluated. In this instance, every row has a probability of 0.5 (or 50\%)
to be retreived.

This query is, however, erroneous due to that the function itself is not
uniformly implemented across WebSQL systems. Using Firefox, the following
value is returned as a result: \\
\begin{centering}
\texttt{Undefined function 'RANDOM' in expression.}
\end{centering}\\
whilst in Google Chrome, the browser generates an alert-window with the
following message:\\
\begin{centering}
\texttt{Error 1: could not prepare statement (1 not authorized to use function: random)}
\end{centering}\\

\paragraph{Select all customers that have placed at least 5 orders}

\begin{lstlisting}[label={sqlstatement4}]
SELECT DISTINCT *
FROM Customers c
LEFT JOIN Orders o
ON c.CustomerID = o.CustomerID
  WHERE COUNT(*) > 4
GROUP BY c.CustomerID 
\end{lstlisting}

This query is designed to retrieve a list of all customers
that have placed more than four orders. At first glance, the
query might seem correct when it, in fact, contains a (sometimes)
hard-to-find error.

As mentioned in the second fully functional query, \texttt{WHERE}
is a filter that is applied to the result \textit{during} retrieval
of the results and \texttt{COUNT} is an aggregate value calculated
from the retrieved result-set. Thus the issue with this query is
that it tries to filter the results using a not yet calculated
value (in this case the number of rows grouped by \textbf{CustomerID}).

This results in the following error message, using Google Chrome:\\
\begin{centering}
\texttt{Error 1: could not prepare statement (1 misuse of aggregate function COUNT())}
\end{centering}

