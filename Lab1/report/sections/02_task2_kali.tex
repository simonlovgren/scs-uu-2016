% Name of section is start of new part of assignment
\pagebreak
\section{Task 2: Get started with Kali Linux and Juice Shop}
This section follows the exercises \textit{Bash Line Commands \& Shell
  Scripting} from Dans Courses\cite{DANSCOURSES} using Kali Linux\footnote{
  \texttt{Kali Linux} is a Linux-distribution, based on the distribution
called Debian\cite{KALIDEBIAN}, geared towards penetration testing and
information security.
}.

Note that in all of the command-line snippets below, the rows with
a leading \texttt{\#} is input and the other rows are the resulting
output (if not stated otherwise).

%%%%%%%%%%%%%%%%%%%%%%%%%%%%%%%%%%%%%%%%%%%%%%%%%%%%%%%%%%
\subsection{Part A: Exercise 1 - Scan a network}
This exercise is aimed towards scanning a network using the
\texttt{nmap} tool. Each of the steps in the exercise will
be briefly explained.

\subsubsection{Part A1: IP-range 192.168.56.1-254}
\label{sec:parta1}
First, \texttt{nmap} is used to scan the network for live
hosts and to scan the live hosts for open ports. Here
the IP-range 192.168.56.1, 192.168.56.2, \dots, 192.168.56.254
is scanned.

\begin{lstlisting}[numbers=none, language={}, frame=single, framexleftmargin={0.2em}]
# nmap 192.168.56.1-254

Starting Nmap 7.01 ( https://nmap.org ) at 2016-10-27 20:01 CEST
Nmap scan report for 192.168.56.1
Host is up (0.00024s latency).
Not shown: 999 closed ports
PORT   STATE SERVICE
80/tcp open  http
MAC Address: 0A:00:27:00:00:00 (Unknown)

Nmap scan report for 192.168.56.100
Host is up (0.00026s latency).
All 1000 scanned ports on 192.168.56.100 are filtered
MAC Address: 08:00:27:32:08:88 (Oracle VirtualBox virtual NIC)

Nmap scan report for 192.168.56.102
Host is up (0.0000040s latency).
Not shown: 999 closed ports
PORT     STATE SERVICE
3000/tcp open  ppp

Nmap done: 254 IP addresses (3 hosts up) scanned in 16.36 seconds
\end{lstlisting}

To not have to scan the network every time the result is processed,
the output can be written to a file. In this case the result is written
to the file \texttt{exercise\_a1.txt}.

\begin{lstlisting}[numbers=none, language={}, frame=single, framexleftmargin={0.2em}]
# nmap 192.168.56.1-254 > exercise_a1.txt
\end{lstlisting}

\pagebreak
To read the contents of a file, the command \texttt{cat} can be used.
It simply prints the contents of a file.

\begin{lstlisting}[numbers=none, language={}, frame=single, framexleftmargin={0.2em}]
# cat exercise_a1.txt

Starting Nmap 7.01 ( https://nmap.org ) at 2016-10-27 20:01 CEST
Nmap scan report for 192.168.56.1
Host is up (0.00024s latency).
Not shown: 999 closed ports
PORT   STATE SERVICE
80/tcp open  http
MAC Address: 0A:00:27:00:00:00 (Unknown)

Nmap scan report for 192.168.56.100
Host is up (0.00026s latency).
All 1000 scanned ports on 192.168.56.100 are filtered
MAC Address: 08:00:27:32:08:88 (Oracle VirtualBox virtual NIC)

Nmap scan report for 192.168.56.102
Host is up (0.0000040s latency).
Not shown: 999 closed ports
PORT     STATE SERVICE
3000/tcp open  ppp

Nmap done: 254 IP addresses (3 hosts up) scanned in 16.36 seconds
\end{lstlisting}

To filter rows containing specific keywords, \texttt{grep} can be used.
In this case a simple keyword is used (report), but depending on implementation
\texttt{grep} can be used with extended regular expressions for powerful
filtration of text and data.

\begin{lstlisting}[numbers=none, language={}, frame=single, framexleftmargin={0.2em}]
# cat exercise_a1.txt | grep report
Nmap scan report for 192.168.56.1
Nmap scan report for 192.168.56.100
Nmap scan report for 192.168.56.102
\end{lstlisting}

To extract a specific part of each line returned by grep, the
\texttt{cut} command is used. In this case, \texttt{cut} is used to split each
line by a delimiting character (here whitespace) and from the
resulting split, the fifth field is extracted.

\begin{lstlisting}[numbers=none, language={}, frame=single, framexleftmargin={0.2em}]
cat exercise_a1.txt | grep report | cut -d " " -f 5
192.168.56.1
192.168.56.100
192.168.56.102
\end{lstlisting}

If the results are unordered, they can be sorted using the command
\texttt{sort}. This command is also capable of removing duplicate
rows, which can be useful for extracting lists containing unique
entries (for example a list of IP addresses).

\begin{lstlisting}[numbers=none, language={}, frame=single, framexleftmargin={0.2em}]
cat exercise_a1.txt | grep report | cut -d " " -f 5 | sort
192.168.56.1
192.168.56.100
192.168.56.102
\end{lstlisting}

\pagebreak
These commands can then be used to filter- and cut data in multiple steps.
In this example, \texttt{grep} matches either \textit{report} or \textit{open}.
\texttt{cut} is then used to, in a systematic way, split each returned row and extract
only the IP-address and open ports of each host on the network.

\begin{lstlisting}[numbers=none, language={}, frame=single, framexleftmargin={0.2em}]
cat exercise_a1.txt | grep "report\|open" | cut -d "/" -f1 | cut -d "r" -f 4 | cut -d " " -f2
192.168.56.1
80
192.168.56.100
192.168.56.102
3000
\end{lstlisting}

\subsubsection{Part A2: IP-range 10.0.1.1-254}
This exercise is analogous to the exercise described in section \ref{sec:parta1},
as it is the same exercise repeated with another IP-range/network containing
more hosts than the network scanned in section \ref{sec:parta1}. Thus the description of
the steps in this section will be even briefer.

The network is scanned using \texttt{nmap} to see whether any hosts are
live and reachable.

\begin{lstlisting}[numbers=none, language={}, frame=single, framexleftmargin={0.2em}]
# nmap 10.0.1.1-254

Starting Nmap 7.01 ( https://nmap.org ) at 2016-10-27 21:37 CEST
Nmap scan report for 10.0.1.2
Host is up (0.011s latency).
Not shown: 997 closed ports
PORT    STATE SERVICE
22/tcp  open  ssh
80/tcp  open  http
443/tcp open  https

Nmap scan report for 10.0.1.30
Host is up (0.0074s latency).
Not shown: 992 filtered ports
PORT    STATE  SERVICE
20/tcp  closed ftp-data
21/tcp  open   ftp
22/tcp  open   ssh
25/tcp  open   smtp
80/tcp  open   http
110/tcp open   pop3
143/tcp open   imap
443/tcp closed https

Nmap scan report for 10.0.1.35
Host is up (0.010s latency).
Not shown: 996 closed ports
PORT    STATE SERVICE
21/tcp  open  ftp
22/tcp  open  ssh
80/tcp  open  http
631/tcp open  ipp

Nmap scan report for 10.0.1.41
Host is up (0.011s latency).
Not shown: 996 closed ports
PORT      STATE SERVICE
21/tcp    open  ftp
22/tcp    open  ssh
631/tcp   open  ipp
13782/tcp open  netbackup

Nmap scan report for 10.0.1.50
Host is up (0.011s latency).
Not shown: 994 closed ports
PORT     STATE SERVICE
22/tcp   open  ssh
53/tcp   open  domain
111/tcp  open  rpcbind
2121/tcp open  ccproxy-ftp
4444/tcp open  krb524
6667/tcp open  irc

Nmap scan report for 10.0.1.55
Host is up (0.011s latency).
Not shown: 999 closed ports
PORT     STATE SERVICE
2121/tcp open  ccproxy-ftp

Nmap scan report for 10.0.1.60
Host is up (0.011s latency).
Not shown: 997 closed ports
PORT     STATE SERVICE
139/tcp  open  netbios-ssn
445/tcp  open  microsoft-ds
1099/tcp open  rmiregistry

Nmap scan report for 10.0.1.65
Host is up (0.011s latency).
Not shown: 991 closed ports
PORT      STATE SERVICE
135/tcp   open  msrpc
139/tcp   open  netbios-ssn
445/tcp   open  microsoft-ds
49152/tcp open  unknown
49153/tcp open  unknown
49154/tcp open  unknown
49155/tcp open  unknown
49156/tcp open  unknown
49157/tcp open  unknown

Nmap scan report for 10.0.1.70
Host is up (0.011s latency).
Not shown: 995 closed ports
PORT     STATE SERVICE
135/tcp  open  msrpc
139/tcp  open  netbios-ssn
445/tcp  open  microsoft-ds
1025/tcp open  NFS-or-IIS
1027/tcp open  IIS

Nmap scan report for 10.0.1.100
Host is up (0.010s latency).
Not shown: 994 closed ports
PORT     STATE SERVICE
22/tcp   open  ssh
80/tcp   open  http
111/tcp  open  rpcbind
443/tcp  open  https
631/tcp  open  ipp
3306/tcp open  mysql

Nmap scan report for 10.0.1.103
Host is up (0.0093s latency).
Not shown: 997 closed ports
PORT     STATE SERVICE
139/tcp  open  netbios-ssn
445/tcp  open  microsoft-ds
2121/tcp open  ccproxy-ftp

Nmap done: 254 IP addresses (11 hosts up) scanned in 21.38 seconds
\end{lstlisting}

The scan is repeated, but the result is stored in a file
(\texttt{scan.txt}) for faster processing.

\begin{lstlisting}[numbers=none, language={}, frame=single, framexleftmargin={0.2em}]
# nmap 10.0.1.1-254 > scan.txt
\end{lstlisting}

With cat, the contents of the file is verified.

\begin{lstlisting}[numbers=none, language={}, frame=single, framexleftmargin={0.2em}]
# cat scan.txt

Starting Nmap 7.01 ( https://nmap.org ) at 2016-10-27 21:37 CEST
Nmap scan report for 10.0.1.2
Host is up (0.0088s latency).
Not shown: 997 closed ports
PORT    STATE SERVICE
22/tcp  open  ssh
80/tcp  open  http
443/tcp open  https

Nmap scan report for 10.0.1.30
Host is up (0.010s latency).
Not shown: 992 filtered ports
PORT    STATE  SERVICE
20/tcp  closed ftp-data
21/tcp  open   ftp
22/tcp  open   ssh
25/tcp  open   smtp
80/tcp  open   http
110/tcp open   pop3
143/tcp open   imap
443/tcp closed https

Nmap scan report for 10.0.1.35
Host is up (0.010s latency).
Not shown: 996 closed ports
PORT    STATE SERVICE
21/tcp  open  ftp
22/tcp  open  ssh
80/tcp  open  http
631/tcp open  ipp

Nmap scan report for 10.0.1.41
Host is up (0.010s latency).
Not shown: 996 closed ports
PORT      STATE SERVICE
21/tcp    open  ftp
22/tcp    open  ssh
631/tcp   open  ipp
13782/tcp open  netbackup

Nmap scan report for 10.0.1.50
Host is up (0.0080s latency).
Not shown: 994 closed ports
PORT     STATE SERVICE
22/tcp   open  ssh
53/tcp   open  domain
111/tcp  open  rpcbind
2121/tcp open  ccproxy-ftp
4444/tcp open  krb524
6667/tcp open  irc

Nmap scan report for 10.0.1.55
Host is up (0.0096s latency).
Not shown: 999 closed ports
PORT     STATE SERVICE
2121/tcp open  ccproxy-ftp

Nmap scan report for 10.0.1.60
Host is up (0.0094s latency).
Not shown: 997 closed ports
PORT     STATE SERVICE
139/tcp  open  netbios-ssn
445/tcp  open  microsoft-ds
1099/tcp open  rmiregistry

Nmap scan report for 10.0.1.65
Host is up (0.0084s latency).
Not shown: 991 closed ports
PORT      STATE SERVICE
135/tcp   open  msrpc
139/tcp   open  netbios-ssn
445/tcp   open  microsoft-ds
49152/tcp open  unknown
49153/tcp open  unknown
49154/tcp open  unknown
49155/tcp open  unknown
49156/tcp open  unknown
49157/tcp open  unknown

Nmap scan report for 10.0.1.70
Host is up (0.0091s latency).
Not shown: 995 closed ports
PORT     STATE SERVICE
135/tcp  open  msrpc
139/tcp  open  netbios-ssn
445/tcp  open  microsoft-ds
1025/tcp open  NFS-or-IIS
1027/tcp open  IIS

Nmap scan report for 10.0.1.100
Host is up (0.011s latency).
Not shown: 994 closed ports
PORT     STATE SERVICE
22/tcp   open  ssh
80/tcp   open  http
111/tcp  open  rpcbind
443/tcp  open  https
631/tcp  open  ipp
3306/tcp open  mysql

Nmap scan report for 10.0.1.103
Host is up (0.0099s latency).
Not shown: 997 closed ports
PORT     STATE SERVICE
139/tcp  open  netbios-ssn
445/tcp  open  microsoft-ds
2121/tcp open  ccproxy-ftp

Nmap done: 254 IP addresses (11 hosts up) scanned in 21.51 seconds
\end{lstlisting}

Using \texttt{grep} the same way as in section \ref{sec:parta1}, the
IP-addresses of the results can be found within the file.

\begin{lstlisting}[numbers=none, language={}, frame=single, framexleftmargin={0.2em}]
# cat scan.txt | grep "report"
Nmap scan report for 10.0.1.2
Nmap scan report for 10.0.1.30
Nmap scan report for 10.0.1.35
Nmap scan report for 10.0.1.41
Nmap scan report for 10.0.1.50
Nmap scan report for 10.0.1.55
Nmap scan report for 10.0.1.60
Nmap scan report for 10.0.1.65
Nmap scan report for 10.0.1.70
Nmap scan report for 10.0.1.100
Nmap scan report for 10.0.1.103
\end{lstlisting}

And using \texttt{cut}, only the IP-addresses are extracted from each
line.

\begin{lstlisting}[numbers=none, language={}, frame=single, framexleftmargin={0.2em}]
# cat scan.txt | grep "report" | cut -d " " -f 5
10.0.1.2
10.0.1.30
10.0.1.35
10.0.1.41
10.0.1.50
10.0.1.55
10.0.1.60
10.0.1.65
10.0.1.70
10.0.1.100
10.0.1.103
\end{lstlisting}

Using sort, the IP-addresses are sorted. In this case, note that
addresses 10.0.1.100 and 10.0.1.103 are sorted before 10.0.1.2 as
the leading digit of the last segment is one.

\begin{lstlisting}[numbers=none, language={}, frame=single, framexleftmargin={0.2em}]
# cat scan.txt | grep "report" | cut -d " " -f 5 | sort
10.0.1.100
10.0.1.103
10.0.1.2
10.0.1.30
10.0.1.35
10.0.1.41
10.0.1.50
10.0.1.55
10.0.1.60
10.0.1.65
10.0.1.70
\end{lstlisting}

Here \texttt{grep} is used to filter lines containing the keyword
\textit{report} or \textit{open}. \texttt{cut} is then systematically used
to extract only IP-addresses and open port numbers.

\begin{lstlisting}[numbers=none, language={}, frame=single, framexleftmargin={0.2em}]
# cat scan.txt | grep "report\|open" | cut -d "/" -f 1 | cut -d "r" -f 4 | cut -d " " -f 2
10.0.1.2
22
80
443
10.0.1.30
21
22
25
80
110
143
10.0.1.35
21
22
80
631
10.0.1.41
21
22
631
13782
10.0.1.50
22
53
111
2121
4444
6667
10.0.1.55
2121
10.0.1.60
139
445
1099
10.0.1.65
135
139
445
49152
49153
49154
49155
49156
49157
10.0.1.70
135
139
445
1025
1027
10.0.1.100
22
80
111
443
631
3306
10.0.1.103
139
445
2121
\end{lstlisting}

\subsection{Part B: Exercise 2 - Sort and filter subdomains \& servers}
\label{sec:yahoo}
In this exercise a more extensive source is used for \texttt{grep}ing
and \texttt{cut}ting text, namely the front page of yahoo.com.

Note, however, that a lot has happened to the website (and source code)
since the exercises were written (last update in September 2013\cite{DANSCOURSES}).
Thus the results of the commands do not fully match the ones in the exercise.

Some characters have also been removed from the source code, as they are incompatible
with latex.

First \texttt{wget} is used to download the source of the website's homepage. It is
stored as \texttt{index.html}.

\begin{lstlisting}[numbers=none, language={}, frame=single, framexleftmargin={0.2em}]
# wget "http://yahoo.com"
\end{lstlisting}

\texttt{cat} is used to verify the contents of the file. The output of the
command has been moved to attachments, as it is rather extensive. See attachment \ref{att:yahooindex}.
\begin{lstlisting}[numbers=none, language={}, frame=single, framexleftmargin={0.2em}]
# cat index.html | grep "yahoo.com"

[output, see attachment]
\end{lstlisting}

\pagebreak
\texttt{grep} is used to filter all rows containing the string \textit{yahoo.com}.

\begin{lstlisting}[numbers=none, language={}, frame=single, framexleftmargin={0.2em}]
# cat index.html | grep "yahoo.com" | cut -d ":" -f 2
    <title>Yahoo</title><meta http-equiv="x-dns-prefetch-control" content="on"><link rel="dns-prefetch" href="//s.yimg.com"><link rel="preconnect" href="//s.yimg.com"><link rel="dns-prefetch" href="//y.analytics.yahoo.com"><link rel="preconnect" href="//y.analytics.yahoo.com"><link rel="dns-prefetch" href="//geo.query.yahoo.com"><link rel="preconnect" href="//geo.query.yahoo.com"><link rel="dns-prefetch" href="//csc.beap.bc.yahoo.com"><link rel="preconnect" href="//csc.beap.bc.yahoo.com"><link rel="dns-prefetch" href="//geo.yahoo.com"><link rel="preconnect" href="//geo.yahoo.com"><link rel="dns-prefetch" href="//comet.yahoo.com"><link rel="preconnect" href="//comet.yahoo.com"><link rel="dns-prefetch" href="//video-api.yql.yahoo.com"><link rel="preconnect" href="//video-api.yql.yahoo.com"><link rel="dns-prefetch" href="//yrtas.btrll.com"><link rel="preconnect" href="//yrtas.btrll.com"><link rel="dns-prefetch" href="//shim.btrll.com"><link rel="preconnect" href="//shim.btrll.com">    <meta http-equiv="Content-Type" content="text/html; charset=utf-8">
url" content="http
//se.yahoo.com/" />    <meta name="referrer" content="origin">        <link href="https
//s.yimg.com/kx/yucs/uh3s/atomic/88/css/atomic-min.css" rel="stylesheet" type="text/css"><link href="https
 {"compr_type"
UH;" class="Ell C-2" href="https
UH;" class="Ell C-2" href="https
UH;" class="Ell C-2" href="https
UH;" class="Ell C-2" href="https
UH;" class="Ell C-2" href="https
uh;src
//se.search.yahoo.com/searchset?pn=" data-ylt-dssbarclose="/" data-ylt-dssbaropen="/" data-ylt-dssstb-link="https
//se.mail.yahoo.com/?src=fpctx"   data-ylk="rspns
//uk.news.yahoo.com/"   data-ylk="rspns
//uk.eurosport.yahoo.com/"   data-ylk="rspns
//www.yahoo.com/style/"   data-ylk="rspns
//www.yahoo.com/beauty/"   data-ylk="rspns
//www.yahoo.com/movies/"   data-ylk="rspns
//www.yahoo.com/tech/"   data-ylk="rspns
//answers.yahoo.com/"   data-ylk="rspns
//se.mobile.yahoo.com/"   data-ylk="rspns
//messenger.yahoo.com/"   data-ylk="rspns
//www.yahoo.com/news/weather/"   data-ylk="rspns
//se.everything.yahoo.com/"   data-ylk="rspns
55
//mobile.yahoo.com/yahoo/?src=gta" data-ylk="t1
//mobile.yahoo.com"><img src="https
//www.yahoo.com/news/weather/se/uppsala-lan/uppsala-24608353/"   data-ylk="itc
//www.yahoo.com/news/weather/se/uppsala-lan/uppsala-24608353/"   data-ylk="itc
//www.yahoo.com/news/weather/se/uppsala-lan/uppsala-24608353/"   data-ylk="itc
//www.yahoo.com/news/weather/se/uppsala-lan/uppsala-24608353/"   data-ylk="itc
//www.yahoo.com/news/weather/se/uppsala-lan/uppsala-24608353/"   data-ylk="itc
//www.yahoo.com/news/weather">Se mer </a>
//info.yahoo.com/privacy/se/yahoo/" class="" data-ylk="t1
 [ { "html"
0,"usePE"
{"td-applet-stream"
{"td-applet-weather"
 'https
 "https
//sb.scorecardresearch.com/p?c1=2&c2=7241469&c7=https%3A%2F%2Fse.yahoo.com%2F&c5=2142872160&cv=2.0&cj=1" />
            <!-- Via https/1.1 media-router-fp8.prod.media.ir2.yahoo.com[BC7D50F2] (YahooTrafficServer) -->
59
\end{lstlisting}

The result is further refined by using \texttt{cut} a second time.

\begin{lstlisting}[numbers=none, language={}, frame=single, framexleftmargin={0.2em}]
# cat index.html | grep "yahoo.com" | cut -d ":" -f 2 |  cut -d "/" -f 3

url" content="http
se.yahoo.com
s.yimg.com
 {"compr_type"
UH;" class="Ell C-2" href="https
UH;" class="Ell C-2" href="https
UH;" class="Ell C-2" href="https
UH;" class="Ell C-2" href="https
UH;" class="Ell C-2" href="https
uh;src
se.search.yahoo.com
se.mail.yahoo.com
uk.news.yahoo.com
uk.eurosport.yahoo.com
www.yahoo.com
www.yahoo.com
www.yahoo.com
www.yahoo.com
answers.yahoo.com
se.mobile.yahoo.com
messenger.yahoo.com
www.yahoo.com
se.everything.yahoo.com
55
mobile.yahoo.com
mobile.yahoo.com"><img src="https
www.yahoo.com
www.yahoo.com
www.yahoo.com
www.yahoo.com
www.yahoo.com
www.yahoo.com
info.yahoo.com
 [ { "html"
0,"usePE"
{"td-applet-stream"
{"td-applet-weather"
 'https
 "https
sb.scorecardresearch.com

59
\end{lstlisting}

Then sort is used with the \texttt{-u} flag, which only returns unique rows.

\begin{lstlisting}[numbers=none, language={}, frame=single, framexleftmargin={0.2em}]
# cat index.html | grep "yahoo.com" | cut -d ":" -f 2 |  cut -d "/" -f 3 | sort -u

0,"usePE"
55
59
answers.yahoo.com
 {"compr_type"
 [ { "html"
 'https
 "https
info.yahoo.com
messenger.yahoo.com
mobile.yahoo.com
mobile.yahoo.com"><img src="https
sb.scorecardresearch.com
se.everything.yahoo.com
se.mail.yahoo.com
se.mobile.yahoo.com
se.search.yahoo.com
se.yahoo.com
s.yimg.com
{"td-applet-stream"
{"td-applet-weather"
UH;" class="Ell C-2" href="https
uh;src
uk.eurosport.yahoo.com
uk.news.yahoo.com
url" content="http
www.yahoo.com
\end{lstlisting}

The result is then further refined by removing trailing text for some links.

\begin{lstlisting}[numbers=none, language={}, frame=single, framexleftmargin={0.2em}]
# cat index.html | grep "yahoo.com" | cut -d ":" -f 2 |  cut -d "/" -f 3 | sort -u | cut -d "\"" -f 1

0,
55
59
answers.yahoo.com
 {
 [ {
 'https

info.yahoo.com
messenger.yahoo.com
mobile.yahoo.com
mobile.yahoo.com
sb.scorecardresearch.com
se.everything.yahoo.com
se.mail.yahoo.com
se.mobile.yahoo.com
se.search.yahoo.com
se.yahoo.com
s.yimg.com
{
{
UH;
uh;src
uk.eurosport.yahoo.com
uk.news.yahoo.com
url
www.yahoo.com
\end{lstlisting}

To continue processing the file, the current output is stored to
a file as the current command is rather long.

\begin{lstlisting}[numbers=none, language={}, frame=single, framexleftmargin={0.2em}]
# cat index.html | grep "yahoo.com" | cut -d ":" -f 2 |  cut -d "/" -f 3 | sort -u | cut -d "\"" -f 1 > domains.txt
\end{lstlisting}

The new file, \texttt{domains.txt}, is further processed as to
extract only domains- and subdomains to yahoo.com using grep
and sort.

\begin{lstlisting}[numbers=none, language={}, frame=single, framexleftmargin={0.2em}]
# cat domains.txt | sort -u | grep "yahoo.com"
answers.yahoo.com
info.yahoo.com
messenger.yahoo.com
mobile.yahoo.com
se.everything.yahoo.com
se.mail.yahoo.com
se.mobile.yahoo.com
se.search.yahoo.com
se.yahoo.com
uk.eurosport.yahoo.com
uk.news.yahoo.com
www.yahoo.com
\end{lstlisting}

The result is then stored in a text file.

\begin{lstlisting}[numbers=none, language={}, frame=single, framexleftmargin={0.2em}]
# cat domains.txt | sort -u | grep "yahoo.com" > domains2.txt
\end{lstlisting}

\pagebreak
\subsection{Part C: Exercise 3 - Bash scripts}
Exercise 3 targets writing bash shell-scripts to automate otherwise tedious
tasks, such as resolving IP-addresses from multiple hostnames or pinging a
wide range of IP-addresses.

Shell-scripts are files containing command-line instructions (or \textit{commands}),
the very same typed into the terminal. In its simplest, a shell-script could be used to alias an
otherwise long commands. Consider the following command, which list the five last log
messages from Google Crome:

$$\texttt{sudo dmesg | grep Chrome | tail -n 5}$$

Instead of writing the full instruction every time, it can be placed in a plain
text file with the file-extension \texttt{.sh}. This file (shell-script) can
then be called directly from the command-line -- as long as it has been allowed
to be executed using, for example, \texttt{chmod +x myfile.sh}.

The shell-script can then be executed by typing \texttt{/path/to/myscipt.sh} into the
terminal\footnote{dynamic paths also work (ex. \texttt{../myscript.sh})}. However, if the script file has been placed in one of the
binary folders\footnote{\texttt{/bin}, \texttt{/usr/local/bin}, \texttt{\textasciitilde/bin}
  or alike, depending on system.}, then the script can be called from anywhere only typing
its name (ex. \texttt{myscript.sh}).

Below are the two shell-scripts, fully commented for every line as to
clearly describe exactly what the script does.

\pagebreak
\subsubsection{Bash script: Resolving hosts from file}
\texttt{ips.sh} is a script that reads hostnames line-by-line from a plain text file,
passing them one-by-one to the command \texttt{host} which resolves the IP-address of
a host. The file is supplied as an argument when calling the script, thus it is not
limited to a specific file.

If the script is called without a filename as argument, it will print an error-text
as well information about how the script is to be used. This is also true for if the name of a
non-existing file has been supplied.

\lstinputlisting[language=bash, caption = \texttt{ips.sh}]{../exercise3/ips.sh}

\pagebreak
The resulting output of running the script with the file \texttt{domains2.txt}
generated in section \ref{sec:yahoo} is the following:

\lstinputlisting[numbers=none, language={}, frame=single, framexleftmargin={0.2em}]{../domain-resolve.txt}

\pagebreak
\subsubsection{Bash script: Ping IP-range}
\texttt{bruteping.sh} is a script that takes an IP-address or IP-range as an
argument and then (sequentially) pings each address. It verifies the format of the
argument using regular expressions to match an IP address. \textbf{Note, however,
that it does not verify that each part of the IP is within the available range (0-255)}.

As with \texttt{ips.sh}, \texttt{bruteping.sh} prints an error-message and how the script
is used if the argument is invalid or none has been provided.

\lstinputlisting[language=bash, caption = \texttt{bruteping.sh}]{../exercise3/bruteping.sh}

This is the result of an example execution:

\lstinputlisting[numbers=none, language={}, frame=single, framexleftmargin={0.2em}]{../ping.txt}
