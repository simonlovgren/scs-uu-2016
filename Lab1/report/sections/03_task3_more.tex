% Name of section is start of new part of assignment
\section{Task 3: Find out more}
There is much more to learn regarding information security. Two examples
are \textit{Metasploit} and \textit{CVE}.

%%%%%%%%%%%%%%%%%%%%%%%%%%%%%%%%%%%%%%%%%%%%%%%%%%%%%%%%%%
\subsection{Part A: Metasploit}
Metasploit is a penetration testing framework, first developed in perl and
released in 2003 with a total of 11 exploits\cite{NOSTARCHMETASPLOIT}. It has
since grown to be one of the world'd most used penetration testing frameworks
\cite{METASPLHOME}.

The framework provides an infrastructure for automating routine- and complex
tasks, allowing the user to concentrate more on the unique aspects of the
penetration testing. The framework does not only come with a set of exploits,
ready-to-use, but also enables the user to relatively easily write custom
exploits that can be added to the collection.

Metasploit is currently available in several versions, with the framework
itself being Open Source.\cite{METASPLEDIT}

%%%%%%%%%%%%%%%%%%%%%%%%%%%%%%%%%%%%%%%%%%%%%%%%%%%%%%%%%%
\subsection{Part B: CVE-2015-3860}
CVE (Common Vulnerabilities and Exposures) is a dictionary of publicly
known information security vulnerabilities and exposures, upheld by
MITRE\cite{CVE}. These vulnerabilities and exposures are assigned a
name of the form CVE-YYYY-NNNN, where YYYY is the year the vulnerability
was assigned a CVE-name and NNNN is an arbitrary digit to distinguish
vulnerabilities from each other. Since 2014, the arbitrary number is no
longer limited to four digits.\cite{CVESYNTAX}

\subsubsection{CVE-2015-3860}
\textbf{CVE-2015-3860} is one of the vulnerabilities stored in the CVE
dictionary, and it refers to a vulnerability found in Android 5.x
(before 5.1.1).

The vulnerability is connected to the lock-screen, where the number of
characters in the passwordEntry input field was not restricted, allowing
attackers to bypass access restrictions by entering a long enough string
that triggers a SystemUI crash. An attacker must be in close proximity
to the device in order to exploit this vulnerability.\ref{CVE-2015-3860}
