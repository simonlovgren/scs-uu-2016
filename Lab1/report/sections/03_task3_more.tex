% Name of section is start of new part of assignment
\section{Task 3: Find out more}
There is much more to learn regarding information security. Two examples
are \textit{Metasploit} and \textit{CVE}.

%%%%%%%%%%%%%%%%%%%%%%%%%%%%%%%%%%%%%%%%%%%%%%%%%%%%%%%%%%
\subsection{Part A: Metasploit}
Metasploit is a penetration testing framework, first developed in perl and
released in 2003 with a total of 11 exploits\cite{NOSTARCHMETASPLOIT}. It has
since grown to be one of the world'd most used penetration testing frameworks
\cite{METASPLHOME}.

The framework provides an infrastructure for automating routine- and complex
tasks, allowing the user to concentrate more on the unique aspects of the
penetration testing. The framework does not only come with a set of exploits,
readu-to-use, but also enables the user to relatively easily write custom
exploits that can be added to the collection.

Metasploit is currently available in several versions, with the framework
itself being Open Source.\cite{METASPLEDIT}

%%%%%%%%%%%%%%%%%%%%%%%%%%%%%%%%%%%%%%%%%%%%%%%%%%%%%%%%%%
\subsection{Part B: CVE-2015-3860}
What is CVE and the notation 'CVE-2015-3860'.
What is specifically CVE-2015-3860.
